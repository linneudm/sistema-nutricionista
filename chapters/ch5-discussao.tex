\chapter{Discussão} \label{ch:discussao}




Com base no levantamento de requisitos, representado pelas histórias de usuário, foi desenvolvido o sistema. Foi feito uma avaliação do sistema sob os critérios de usabilidade, que deve contemplar os seguintes fundamentos: facilidade de uso, facilidade de aprendizado, memorabilidade, produtividade, flexibilidade e satisfação do usuário. Abaixo segue uma tabela com os critérios de usabilidade segundo \citeonline{nogueira}, ao qual sob ele foi formulado o questionário.
O formulário foi criado através do \textit{Google Forms} onde foi possível gerar um questionário que aborde questões que assegurem a IHC (interface humano-computador).

\begin{table}[hbt]
\centering
\begin{tabular}{p{5cm}|p{10cm}}
\hline
\textbf{Critérios de Usabilidade} & \multicolumn{1}{c}{\textbf{Formas de Aferição}}                                                                                                                                                                    \\ \hline
Facilidade de uso                 & Mensurar,a,velocidade,e,a,quantidade,de,erros,durante,a,
execução,de,determinada,tarefa,,que,caso ocorram, devem ser facilmente recuperados; (PREECE, 1994) (NIELSEN, 1993) (ISO 9241-11, 1998) \\ \hline
Facilidade de aprendizado         & Mensurar,o,tempo,e,o,esforço,necessários,para,que,os,
usuários,tenham,um,determinado,padrão,de desempenho; (NIELSEN, 1993) (PREECE, 1994)                                                       \\ \hline
Satisfação do usuário             & Avaliar se o usuário gosta do sistema e sente prazer em trabalhar com ele; (NIELSEN, 1993) (PREECE, 1994) (ISO 9241-11, 1998)                                                                  \\ \hline
Produtividade                     & Mensurar,o,ganho,de,produtividade,do,usuário,ao,
aprender,a,
utilizar,o,sistema,proposto;(NIELSEN, 1993) (ISO 9241-11, 1998)                                                                     \\ \hline
Flexibilidade                     & Avaliar o nível de customização e personalização da interface pelo usuário; (PREECE, 1994)                                                                                                     \\ \hline
Memorabilidade                    & Avaliar,o,nível,de,treinamento,necessário,para,reciclar,
usuários,eventuais,do,sistema. (NIELSEN, 1993)                                                                                         \\ \hline
\end{tabular}
\label{tableNogueira} 
\caption{Critérios de Usabilidade \cite{nogueira}}
\end{table}

Feito a pesquisa através do \textit{Google Forms}, foi obtido os resultados, apresentados a seguir. A pesquisa foi feita com nutricionistas que já faziam o uso do sistema e após o uso, foi solicitado que respondessem ao questionário. O sistema é totalmente gratuito e está disponível no \textit{link}: https://sysnut.herokuapp.com, possuindo apenas limite de acessos mensais, por motivo de seu domínio e hospedagem serem gratuitos.

