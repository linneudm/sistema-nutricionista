\chapter{Referencial Teórico} \label{ch:referencial}

{\color{red}
Este capítulo apresnta o referencial teórico utilizado para construção dos systema Web proposto (\SysNut). Vale ressaltar que somente os conceitos técnicos referentes à área de Nutrição são apresentados aqui. Detalhes sobre a implementação e ferramentas utilizadas serão apresentados no Capítulo \ref{ch:sysnut}.
}

\section{Atividades do Nutricionista Apoiadas Pelo \SysNut}

{\color{red}
\subsection{Avaliação X}
\subsection{Avaliação Y}
\subsection{Acompanhamento Z}

\section{Conceitos Importantes}

Esta seção aprenta conceitos importantes existentes no contexto das atividades apresentadas na seção anterior\ldots
}


\subsection{Taxa Metabólica Basal (TMB)}

A taxa metabólica basal (TMB) é a quantidade de energia necessária para a
conservação das funções vitais do organismo, representando a maior parte do
consumo energético diário em humanos (cerca de 50\% a 70\%). Sendo calculada em
condições padrão de jejum, repouso físico e mental em ambiente calmo com controle
de temperatura, iluminação e barulhos \cite{ruiz} \cite{harris1}.

A TMB sofre grande influência da massa magra, sexo, idade, composição
corporal e predisposição genética. Fatores como o funcionamento do sistema nervoso
e os hormônios tireoidianos, também contribuem para diferença da TMB entre os
indivíduos. Para a estimativa da TMB, foram desenvolvidas várias equações
matemáticas, utilizando variáveis de fácil mensuração e de baixo custo, como idade,
altura e massa corporal total. Entre tantas equações, são utilizadas as seguintes:
\citeonline{harris1}, sua reformulação \citeonline{harris2} e \citeonline{cunningham} por possuírem grande
aceitabilidade e credibilidade pelas entidades relacionadas a Nutrição \cite{weijs}.

A Taxa Metabólica Basal torna-se indispensável durante o atendimento nutricional, visto
que ele é responsável pela individualização da sua dieta, de forma que possa atender
todas as necessidades calóricas do mesmo, auxiliando na melhora do quadro
nutricional do paciente \cite{pedrosa}.

A ferramenta criada a partir deste trabalho deve então focar nos dados que estas fórmulas podem apresentar
como resultado, a fim de que possa facilitar o atendimento por parte do nutricionista,
ao tempo que o paciente obtenha um resultado com maior satisfação.


\subsection{Gasto Enérgico Total (GET)}

O gasto energético total (GET) compreende a soma de todos os gastos energéticos diários a seguir: A
taxa metabólica basal (TMB) que compreende o gasto energético necessário para a
consumação das funções vitais do organismo; o gasto energético da atividade física
ou fator de atividade (FA), que representa o gasto calórico com as atividades físicas
do cotidiano e o exercício físico; e o efeito térmico dos alimentos (ETA), relacionado
com a digestão, a absorção e o metabolismo dos alimentos. Em indivíduos saudáveis,
a TMB corresponde aproximadamente de 60\% a 70\% do gasto diário, o ETA entre 5\% 
e 15\% e o GEAF ou FA de 15\% a 30\%, sendo este último o elemento que mais varia
entre os indivíduos \cite{hill}. 

\subsection{TACO - Tabela Brasileira de Composição de Alimentos}

Reconhecida oficialmente pelo Ministério da Saúde e produzida pela Universidade de Campinas, o objetivo da TACO é gerar dados sobre a composição dos principais alimentos consumidos no Brasil, de maneira que possa assegurar a confiabilidade dos resultados \cite{taco}. 
Para desenvolver o sistema, foi necessário utilizar-se de um banco de dados que dispusesse das informações nutricionais necessárias para fazer a anamnese alimentar. Para isto, esta ferramenta foi escolhida. A tabela possui mais de 1000 alimentos cadastrados e informa detalhadamente a composição de cada alimento.
