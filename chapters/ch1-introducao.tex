\chapter{Introdução} 

A alimentação é um dos momentos mais importantes na vida das pessoas tanto
por fatores quanto por fatores sociais, culturais e científicos. É
por meio dela que se obtém a energia necessária para desenvolver atividades diárias
e outros nutrientes primordiais a saúde, como vitaminas e minerais. \cite{proenca}.

O Brasil atualmente passa por uma etapa de transição na alimentação de sua
população, a mesma que antes enfrentava a desnutrição, nos dias atuais apresenta
alta prevalência de obesos, isso pode ser explicado pela mudança drástica na
alimentação. Com o aumento da renda básica das famílias, o \textit{fast-food} tornou-se
presente nas mesas da população brasileira. Alimentos como estes, são ricos em
gordura saturada, açúcar e sódio, os quais são precursores de diversas doenças crônicas não
transmissíveis (DCNT) como, diabetes mellitus tipo II, doenças no sistema
cardiovascular, obesidade mórbida, entre outras \cite{schuster}.
A preocupação com a alimentação e qualidade de vida se faz cada
dia mais presente no cotidiano de todos os brasileiros, decorrência do uso exacerbado
dos \textit{fast-food}. O nutricionista, 
profissional responsável pela prescrição dietoterápica, tem ganho mais cada vez mais espaço no mercado de
trabalho, perante sua importância para manutenção da saúde do ser humano
\cite{brasil}.

Diante da importância do seu trabalho, o nutricionista tem buscado melhorias
no seu atendimento, que por vezes torna-se cansativo perante a vasta quantidade de
cálculos para análise do estado nutricional do paciente, das necessidades energéticas
e para adequação do cardápio. A necessidade de agilizar e aprimorar o atendimento
nutricional acarretou o surgimento de softwares que atendessem as necessidades
básicas como resolução de equações e maior organização dos dados dos pacientes
\cite{vieira}.

A ferramenta proposta neste trabalho, além de ajudar o nutricionista em seu cotidiano de trabalho,
também contribui com a redução do índice de obesidade existente no Brasil, através da utilização da mesma
de forma totalmente gratuita por parte de profissionais da nutrição que acabaram de ingressar no mercado
de trabalho e ainda não possuem nenhuma ferramenta que auxiliem a sua rotina.


\section{Contexto e Problema}

Durante o acompanhamento do paciente, o nutricionista muitas vezes precisa
consultar diversos autores para efetuar um trabalho com mais precisão. Esta etapa do trabalho do profissional
torna-se bastante dificultoso sem o auxílio de uma ferramenta que faça a pesquisa e efetue os cálculos automaticamente.

Para \citeonline{araujo}, antropometria ou avaliação antropométrica é um importante método para avaliação do estado nutricional de um indivíduo e possibilita a detecção de alteração do estado nutricional do mesmo ou até mesmo de coletividades. A Organização Mundial da Saúde (OMS) afirmou em 1995 que a avaliação antropométrica é uma importante ferramenta para detecção para grupos, mas que não deve ser utilizada para diagnóstico final \cite{oms}.

O acompanhamento nutricional individualizado é extremamente importante para
o paciente obter resultados relevantes de acordo com o objetivo traçado pelo
nutricionista, visto que, cada indivíduo possui uma necessidade energética diferente
e a distribuição dos macronutrientes e micronutrientes é realizada de acordo com seu
tipo de organismo, objetivo a ser alcançado e estado nutricional, mediante avaliação
antropométrica.
O nutricionista necessita usar diversas fórmulas
e equações no seu dia a dia, seja para avaliar o estado nutricional do
paciente ou para esquematizar um plano alimentar, fórmulas essas que nem sempre
são sinônimos de praticidade e simplicidade. Esse fator pode trazer perdas
significativas ao profissional da nutrição, visto que são necessários vários cálculos
para prestar atendimento adequado ao seu público-alvo.

Diante disso, é imprescindível uma ferramenta computacional que possibilite a inclusão de refeições, troca de mensagens entre
os usuários do sistema (paciente e nutricionista), 
cálculo das necessidades energéticas e analise do
cardápio auxiliando o profissional durante seu atendimento, trazendo mais
comodidade e rapidez ao seu serviço prestado, de qualquer dispositivo que esteja acessando, desde
que tenha acesso à internet.

\section{Objetivos}

Esse trabalho teve como objeto o desenvolvimento de um sistema para
nutricionistas, voltado para o atendimento nutricional de seus pacientes, atendendo
todas as necessidades primárias de um atendimento individualizado.

\section{Organização do trabalho}
Este trabalho está organizado da seguinte maneira:

{\color{red}

 \begin{itemize}

	\item Capítulo \ref{ch:referencial}  - Descreve descritas todas as ferramentas e técnicas que foram utilizadas durante o desenvolvimento do sistema.
	\item Capítulo 3 – Discute alguns dos trabalhos encontrados que possuem relação com o trabalho do autor.
	\item Capítulo 4 – Explica como o sistema foi implementado e relaciona os casos de uso.
	\item Capítulo 5 – Apresenta e discute os resultados da implementação do projeto.
	\item Capítulo 6 - Aborda os principais achados previstos pelo autor.

\end{itemize}
}